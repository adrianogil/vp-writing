\section{Introdução}

%% - Contextualizar VR

A Realidade Virtual possui um grande atrativo pelo sua forma inovadora no consumo de conteúdo. Imersão é sua característica mais marcante, possibilitando aos usuários um profundo envolvimento com objetos e cenários em 3D dentro de um mundo virtual.

%% - Contextualizar VR Headsets e VR Mobile


%% - Falar sobre a importância do consumo de mídias no VR Mobile
Aplicações de Realidade Virtual permitem uma maior discrição e um maior sensação de interação com conteúdo multimídia. Assim, mesmo fotos e vídeos usuais se tornam uma experiência diferenciada para o usuário VR. Muitas aplicações tem por mote funcionar como galeria de mídia dentro do ambiente de realidade virtual, por exemplo, \textit{VR Gallery} da Samsung é uma delas.

%% - Contextualizar ferramentas de desenvolvimento para VR Mobile (Unity, GVRF)
Unity é um motor de jogos WYSIWYG (What You See Is What You Get) \cite{sv2015popolin}, atualmente é a mais ampla ferramenta para o desenvolvimento de aplicações de realidade virtual mobile, presente como motor de criação em 60\% do conteúdo criado para realidades virtual e aumentada, conta com uma comunidade de desenvolvedores presentes em todo o mundo, além de ser utilizada por grandes corporações como Microsoft e Disney. (https://unity3d.com/pt/public-relations)

%% - Proposta: Desenvolver uma arquitetura de alto desempenho de video player para plataformas VR mobile
Este artigo tem por proposta a implementação de uma arquitetura de alto desempenho para video player em plataformas de realidade virtual e aumentada utilizando dispositivos móveis. Para avaliações desempenhos serão feitas comparações entre dois frameworks de renderização: Unity e GVRF.

%% - Estrutura do Artigo
Na seção \ref{sec-relatedworks} analisamos alguns trabalhos que descrevem o uso de video players no contexto de dispositivos móveis e em realidade virtual e aumentada. A metodologia adotada é descrita na seção \ref{sec-methodology}. A arquitetura implementada foi definda na seção \ref{sec-videoplayer-arch}. Resultados são apresentados na seção \ref{sec-results}. Por fim, as conclusões e perspectivas de trabalhos futuros são mencionadas na seção \ref{sec-conclusion}