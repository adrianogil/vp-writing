\documentclass[12pt]{article}

\usepackage{sbc-template}
\usepackage{graphicx,url}
\usepackage[utf8]{inputenc}
\usepackage[brazil]{babel}

\sloppy

\title{Instructions for Authors of SBC Conferences\\ Papers and Abstracts}

\author{Luciana P. Nedel\inst{1}, Rafael H. Bordini\inst{2}, Flávio Rech
  Wagner\inst{1}, Jomi F. Hübner\inst{3} }


\address{Instituto de Informática -- Universidade Federal do Rio Grande do Sul
  (UFRGS)\\
  Caixa Postal 15.064 -- 91.501-970 -- Porto Alegre -- RS -- Brazil
\nextinstitute
  Department of Computer Science -- University of Durham\\
  Durham, U.K.
\nextinstitute
  Departamento de Sistemas e Computação\\
  Universidade Regional de Blumenal (FURB) -- Blumenau, SC -- Brazil
  \email{\{nedel,flavio\}@inf.ufrgs.br, R.Bordini@durham.ac.uk,
  jomi@inf.furb.br}
}

\begin{document}

\maketitle

\begin{abstract}
  This meta-paper describes the style to be used in articles and short papers
  for SBC conferences. For papers in English, you should add just an abstract
  while for the papers in Portuguese, we also ask for an abstract in
  Portuguese (``resumo''). In both cases, abstracts should not have more than
  10 lines and must be in the first page of the paper.
\end{abstract}

\begin{resumo}
  Este meta-artigo descreve o estilo a ser usado na confecção de artigos e
  resumos de artigos para publicação nos anais das conferências organizadas
  pela SBC. É solicitada a escrita de resumo e abstract apenas para os artigos
  escritos em português. Artigos em inglês deverão apresentar apenas abstract.
  Nos dois casos, o autor deve tomar cuidado para que o resumo (e o abstract)
  não ultrapassem 10 linhas cada, sendo que ambos devem estar na primeira
  página do artigo.
\end{resumo}

\section{Introdução}

%% - Contextualizar VR mobile

A Realidade Virtual possui um grande

%% - Falar sobre a importância do consumo de mídias no VR Mobile
Aplicações são

%% - Contextualizar ferramentas de desenvolvimento para VR Mobile (Unity, GVRF)
Unity é um motor de jogos WYSIWYG (What You See Is What You Get) \cite{sv2015popolin}, atualmente é a mais ampla ferramenta para o desenvolvimento de aplicações de realidade virtual mobile, presente como motor de criação em 60\% do conteúdo criado para realidades virtual e aumentada, conta com uma comunidade de desenvolvedores presentes em todo o mundo, além de ser utilizada por grandes corporações como Microsoft e Disney. (https://unity3d.com/pt/public-relations)

%% - Proposta: Desenvolver uma arquitetura de alto desempenho de video player para plataformas VR mobile

Lorem ipsum dolor sit amet, consectetur adipisicing elit, sed do eiusmod tempor incidi-dunt ut labore et dolore magna aliqua.  Ut enim ad minim veniam, quis nostrud exerci-tation ullamco laboris nisi ut aliquip ex ea commodo consequat.  Duis aute irure dolorin reprehenderit in voluptate velit esse cillum dolore eu fugiat nulla pariatur.  Excepteursint occaecat cupidatat non proident, sunt in culpa qui officia deserunt mollit anim id estlaborum.


%% - Estrutura do Artigo

\section{Trabalhos Relacionados}


%% - Listar trabalhos relacionados a video players para VR Mobile


%% - Listar trabalhos que falem sobre o desempenho de video players no mobile

Em \cite{wild2018inaccessibility}

%% - Listar trabalhos que falem sobre a renderização de videos no VR
MR360 definido em \cite{rhee2017mr360} propõe uma composição de objetos virtuals 3D durante a execução de um video 360. O video panorâmico é usado tanto como fonte de luz para iluminar os objetos virtuais quanto para compor o pano de fundo a ser renderizado.


%% - Listar trabalhos que lidam com o desenvolvimento de plugins Unity Android
Calory Battle AR desenvolvido por \cite{kim2014using} é um jogo de realidade aumentada construído em Unity cujo objetivo é conscientizar o jovem público-alvo sobre os benefícios da alimentação saudável através do estímulo à pratica de exercícios ao explorar o espaço físico em busca do desarmamento das "bombas calóricas".
\cite{sv2015popolin} mostra o desenvolvimento do Unity Cluster Package, voltado para a criação de aplicações de multiprojeção para sistemas baseados em Aglomerado Gráfico (AG) através do modelo Mestre-Escravo no editor Unity.


%% - Referenciar nosso trabalho do SVR sobre o GVRF

Lorem ipsum dolor sit amet, consectetur adipisicing elit, sed do eiusmod
tempor incididunt ut labore et dolore magna aliqua. Ut enim ad minim veniam,
quis nostrud exercitation ullamco laboris nisi utM aliquip ex ea commodo
consequat. Duis aute irure dolor in reprehenderit in voluptate velit esse
cillum dolore eu fugiat nulla pariatur. Excepteur sint occaecat cupidatat non
proident, sunt in culpa qui officia deserunt mollit anim id est laborum.
\section{Arquitetura Proposta}

Lorem ipsum dolor sit amet, consectetur adipisicing elit, sed do eiusmod
tempor incididunt ut labore et dolore magna aliqua. Ut enim ad minim veniam,
quis nostrud exercitation ullamco laboris nisi ut aliquip ex ea commodo
consequat. Duis aute irure dolor in reprehenderit in voluptate velit esse
cillum dolore eu fugiat nulla pariatur. Excepteur sint occaecat cupidatat non
proident, sunt in culpa qui officia deserunt mollit anim id est laborum.

\bibliographystyle{sbc}
\bibliography{sbc-template}

\end{document}
