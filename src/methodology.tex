\section{Metodologia}

%% - Definir tipos de mídias a serem consumidas no VR Mobile
%% - Definir formas de visualização do conteúdo
%% - Comparativo de diferentes implementações (Unity x GVRF)

Em um ambiente virtual, poderemos ter a manipulação de alguns tipos de mídia, tais como: áudio, imagem e vídeo. Dentre estes, a mídia no formato de vídeo foi escolhida para ser o objeto de estudo deste trabalho. Logo, queremos propor uma arquitetura que tenha um ótimo desempenho em uma plataforma VR no ambiente \textit{mobile}.

Assim, a metodologia pode ser dividida nos seguintes passos:

\begin{enumerate}
    \item Definição dos formatos de vídeo a serem experimentados.
    \item Definição das formas de visualização ou renderização do vídeo no ambiente virtual.
    \item Comparativo de duas diferentes implementações no ambiente \textit{mobile}.
\end{enumerate}

Os vídeos escolhidos para experimentar a arquitetura proposta são do formato XYZ, pois é o formato com grande utilização nos dispositivos móveis. Além deles, foram escolhidos alguns vídeos em 360\degree para experimentarmos esta forma de visualização.
%% Obs.: deveremos experimentar carregamento de streams? (videos do youtube, por exemplo)

A renderização do conteúdo do vídeo será visualizada nas seguintes formas: cilindro, esfera, cubo, plano e 360\degree.

Visando testar a arquitetura nos dispositivos móveis, duas plataformas de desenvolvimento foram escolhidas: Unity\footnote{https://unity3d.com/} e GVRF\footnote{http://www.gearvrf.org/}.
%% Quais métricas elas serão comparadas?