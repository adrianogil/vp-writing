\section{Arquitetura}

%% - Poderíamos separar em duas camadas: Plataforma e Renderização.

Buscando a implementação otimizada de um Video Player para plataformas de Realidade Virtual móveis, dividimos nossa arquitetura em duas camadas principais:

\begin{enumerate}
    \item Camada de plataforma: implementação nativa para lidar com operações de I/O para consumo e descompressão de mídias.
    \item Camada de renderização: uso de algum framework de renderização para tornar vísivel dentro de um universo virtual 3D o conteúdo extraído da mídia.
\end{enumerate}

\subsection{Camada de Plataforma}

%% - Na questão de Plataforma, descrevemos um pouco o lado do Android de consumo de mídias (Uso do MediaPlayer, MediaCodec, uso do OpenGL, tipos de textura, ….), interação com o Unity via plugins

A camada de plataforma é responsável pelo consumo de mídias, alocação e gerenciamento de memória para armazenar o conteúdo descomprimido, lidar com operações de I/O.

\subsection{Camada de Renderização}

%% - Na Renderização podemos explorar diferentes formas de visualização do video (em um Quad, numa esfera em 360, em um grid de voxels), descrever shaders